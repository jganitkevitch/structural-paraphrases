\documentclass[11pt]{article}
\usepackage{acl2010}
\usepackage{times}
\usepackage{url}
\usepackage{latexsym}
\usepackage{graphicx}
%\setlength\titlebox{6.5cm}    
% You can expand the title box if you really have to

\title{Learning Structural and Sentential Paraphrases from Parallel Corpora}

\author{Juri Ganitkevitch \and Chris Callison-Burch\\ 
Center for Language and Speech\\ 
Johns Hopkins University}

\date{}

\begin{document}
\maketitle

\begin{abstract}
Later.
\end{abstract}

\section{Introduction} \label{introduction}

Motivate paraphrasing within the field (same meaning, transformations,
entailment).

\section{Related Work} \label{related_work}

There have been many proposed approaches to paraphrase induction, most
of which can be placed in two groups, according to what data they use
to source their method. Monolingual approaches typically make use of
vast amounts of English text to extract paraphrases. Due to the
availability of good syntactic and dependency parsers, these methods
are often able to infer \emph{structural} paraphrases, i.e.\
paraphrastic patterns that capture either syntactic or semantic
information and can be generalize via ``slots''. However, the coverage
of such systems tends to be low.

\ldots

Bilingually sourced approaches, on the other hand make use of the
relative abundance of sentence-parallel corpora and extract bilingual
tables of phrases (cite Chris) or patterns (cite Nitin, Zhao) from
which paraphrases can be extracted by pivoting over the non-English
side of the table. However, due to the nature of bilingual phrase
tables, the resulting paraphrases are often restricted to
surface-level.
\emph{not quite true, since Zhao does extract labled pattern based on
  dependency graphs. need to look into that and categorize, organize
  prior work properly.}


Should distinguish between phrasal and structural paraphrases, as well
as bilingually and monolingually sourced approaches to extraction.

Perhaps: translation as bilingual paraphrasing, and how that thought
brings about the pivot approach.

Reference monolingual approaches that are structural, pivot-based
approaches that are phrasal, Nitin's stuff.

Note how our approach sort-of unifies the two; there's an analogy to
Chris' syntactic constraints as well as the obvious step from Hiero to
SAMT.

\section{Sentential Paraphrasing} \label{sentential_paraphrasing}

move to analysis?

We are interested in sentential paraphrases. Why are we interested?
More powerful than locally constrained, gives us large-scale changes
to sentential structure, which can be cruicial to applications such as
detecting entailment or automatically creating significantly differing
references. \emph{However, phrasal decoder-based approaches should be
  able to achieve similar re-ordering effects (if not the generality,
  which we only implicitely achieve, really). Maybe we should add a
  phrase-based baseline in addition to Hiero? Did Nitin talk about
  this?}

While the definition of a phrasal paraphrase is intuitively clear,
sentential paraphrases are much harder to define. When paraphrasing a
sentence $s$ into a new sentence $t$, the term suggests that we expect
the changes to $s$ to be above a certain threshold for $t$ to be
considered a sentential paraphrase. 

\section{To Add}

Formal: Synchronous grammars, with the usual examples (one phrasal, one
structural).

Formal: Log-linear model for features, weights will be optimized to some
objective, we discuss those in later section.

Diagram for grammar extraction (two sentence pairs with trees and
alignments, show how that gets us a paraphrase pattern). 

\section{Paraphrase Acquisition} \label{acquisition}

The method we present in this work extends Nitin's pivot-based Hiero
paraphrasing approach to the richer, syntax-informed SAMT
formalism. Starting with a bilingual parallel corpus, we use the
familiar MT and parsing (cite!) machinery to word-align the data and
extract an SAMT-style grammar.

Elaborate on this, reference the appropriate work. This is more about
the pivoting than it is about the MT-style application of the
paraphrases.

In Section~\ref{data_collection} we give a brief description of the
data sets used in our experiments and outline the tools used to
process them. Section~\ref{extraction} elaborates on the extraction
of the bilingual translation grammars and their transformation into
monolingual paraphrasing grammars.

\subsection{Paraphrase Grammar Extraction} \label{extraction}

Some talk about the SAMT approach needs to go here. 

We extract SAMT translation grammars for nine languages. Give some
details on the pipeline, mention that the grammars are
\emph{gargantuan} (this word needs to be in the paper!), but that the
whole process is very well-suited for MapReduce (even though, we
didn't use it).

\subsection{Creating Paraphrase Rules} \label{rule_creation}

\subsubsection{Rule Body} \label{rule_body}

To create paraphrase rules from bilingual translation rules, we pivot
over the foreign side of the translation rule, with the additional
constraint that the rule's head, i.e.\ the label that governs the
rule.

Mention the proper mapping and flipping of nonterminals.

\subsubsection{Rule Features} \label{rule_features}

The SAMT grammars our paraphrasing system is based on provide a rich
feature set that takes into account source- and target-side
frequencies, reordering of NTs and lexical translation probablities
for each rule. When transforming the bilingual grammars into a
monolingual paraphrase grammar we preseve the feature set and

Shouldn't give details on every single feature, but point out some key
approaches:
\begin{itemize}
\item probablistic features multiply 
\item target-side indicator features are inherited
\item actually state what happens with other indicator features
  (re-ordering, punctuation, rareness etc)
\end{itemize}


\section{Analysis} \label{analysis}

one of the motivations for our approach was the expectation of being
able to acquire structural paraphrases, especially paraphrases taht
are capable of learning long-distance transformations such as
passivization, dative shift, possessive something and prepositional
paraphrases.

\section{Paraphrasing System} \label{approach}

The paraphrase extraction described in the precious section yields an
English-to-English paraphrase grammar

\subsection{Training}

We use MERT. It's great.

\subsection{Objective Function} \label{features}

Och (MERT) has shown that it is best to tune to the objective function
that will be used for evaluation..

\subsubsection{Paraphrase BLEU}

\subsubsection{Summarization BLEU}
yet to be thought out, but another example here
would be great

get some data for summarization? other task?

\section{Experimental Results} \label{results}


\subsection{Data Collection} \label{data_collection}

We use Europarl v. 5. Align with Berlekey and parse with The Parser.


\subsection{Reference Expansion for SMT} \label{smt_application}

A straightforward way to evaluate a paraphrasing system is by using to
improve an SMT system's performance. Cite Chris and Nitin. Compare to Hiero
baseline.  

\subsection{Possibly Another Paraphrase
  Application} \label{other_application}

Maybe summarization with examples? Is there any ``hard'' eval to be had? Human via MTurk?

\section{Conclusion} \label{conclusion}

Look, we unified everything in the field and made all this stuff from
the previous section much better. Or did we?

\bibliographystyle{acl}
\bibliography{paraphrasing}

\nocite{*}

\end{document}

